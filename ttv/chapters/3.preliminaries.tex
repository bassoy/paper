\section{Background}
\label{sec:preliminaries}

\begin{comment}
\begin{figure}[t]
	\centering
	\subfigure[]{\includegraphics[height=3cm,width=.2\linewidth]{figures/3D-Subdomain-Sliced.eps} }	
	\hspace{1cm}
	\subfigure[]{\includegraphics[height=3cm,width=.2\linewidth]{figures/3D-Subdomain-Strides.eps} }
	\hspace{1cm}
	\subfigure[]{\includegraphics[height=3cm,width=.2\linewidth]{figures/3D-Subdomain-Strides3.eps} }
	\caption{Subtensors with an index offset $f_r > 1$ where (a) is generated with $t_r=1$ for all $r \in \{1,2,3\}$, (b) with $t_1 > 1$ and $t_2 = t_3 = 1$ and (c) with $t_r > 1$ for all $r \in \{1,2,3\}$.}
	\label{fig:fig}
\end{figure}
\end{comment}

\subsubsection{Notation}
%\paragraph{Tensor}
An order-$p$ \tit{tensor} is a $p$-dimensional array or hypermatrix with $p$ modes \cite{lim:2017:hypermatrices}.
For instance, scalars, vectors and matrices are order-$0$, order-$1$ and order-$2$ tensors.
We write $a$, $\mba$, $\mbA$ and $\mubA$ in order to denote scalars, vectors, matrices and tensors. % \cite{lee:2018:fundamental}
In general we will assume the order of a tensor to be $p$ and explicitly mention it otherwise.
Each dimension $n_r$ of the $r$-th mode shall satisfy $n_r>1$.
The $p$-tuple $\mbn$ with $\mbn = (n_1,n_2,\dots,n_p)$ will be referred to as a \tit{dimension tuple}.
We will use round brackets $\mubA(i_1,i_2,\dots,i_p)$ or $\mubA(\mbi)$ together with a multi-index $\mbi = (i_1,i_2,\dots,i_p)$ to identify tensor elements.
The set of all multi-indices of a tensor is denoted by $\mcI$ which is defined as the Cartesian product of all index sets $I_r = \{1,\dots,n_r\}$.
The set $\mcJ = \{0,\dots,\bar{n}-1\}$ contains (relative) memory positions of an order-$p$ tensor with $\bar{n} = n_1 \cdot n_2 \cdots n_p$ contiguously stored elements with $|\mcI| = |\mcJ|$.
A \tit{subtensor} denoted by $\mubA'$ shall reference or view a subset of tensor elements where the references are specified in terms of $p$ index ranges.
The $r$-th \tit{index} \tit{range} shall be given by an index pair denoted by $f_r \colon l_r$ with $1 \leq f_r \leq l_r \leq n_r$.
%The parameter $t_r$ with $t_r > 0$ is the step size or increment of the $r$-th mode.
%For convenience we might also write $\mbr_r$ to denote an index range. 
We will use $\colon_r$ to specify a range with all elements of the $r$-th index set.
% such that $\mbr_r \in I_{r}^{m_r}$.
%An \tit{index} \tit{lattice} $\mbR$ with $\mbR = (\mbr_1, \dots, \mbr_p)$ is then an element of the set $\prod_{r=1}^p I_{1}^{m_r}$.
%We might specify a subtensor by using round brackets $\mubS = \mubA(\mbR)$ which is the same as $\mubS = \mubA(f_1 \colon l_1, \dots, f_p \colon l_p)$ if only index pairs are considered. 
%The index sets $K_r$ and the multi-index set $\mcK$ of a subtensor is defined in analogy to the index and multi-index set of tensor where the $r$-th dimension $m_r$ with $|K_r| = m_r$ is given by $m_r = l_r - f_r + 1$.
A subtensor is an order-$p'$ \tit{slice} if all modes of the corresponding order-$p$ tensor are selected either with a full index range or a single index where $p'$ with $p'\leq p$ is the number of all non-singleton dimensions.
A \tit{fiber} is a tensor slice with only one dimension greater than $1$. 


\subsubsection{Non-Hierarchical Storage Formats and Memory Access}
Given a dense order-$p$ tensor, we use a \tit{layout} \tit{tuple} $\mbpi \in \bbN^p$ to encode non-hierarchical storage formats such as the well known first-order or last-order layout.
They contain permuted tensor modes whose priority is given by their index.
%Given a layout tuple of length $p$ and $k \in \bbN$ with $k \leq p$, 
For $1 \leq k \leq p$, an element $\pi_r$ of \tit{$k$-order} \tit{layout} tuple is defined as $k-r+1$ if $1 < r \leq k$ and $r$ in any other case. 
The well-known first- and last-order storage formats are then given by $\mbpi_F = (1,2,\dots,p)$ and $\mbpi_{L} = (p,p-1,\dots,1)$.
%The set $\{\mbpi \in \bbN^p \mid \pi_r \neq \pi_q \wedge 1 \leq r \neq q \leq p\}$
%Using a permutationelements of an order-$p$ tensor can be stored in $p!$ different ways  non-hierarchical storage formats of an order-$p$ tensor contains 
%Hence, elements of an order-$p$ tensor can be stored in $p!$ different ways.
Given a layout tuple $\mbpi$ with $p$ modes, the $\pi_r$-th element of a \tit{stride} \tit{tuple} is given by
\begin{equation}
\label{equ:stride.tuple}
w_{\pi_r} = n_{\pi_{1}} \cdot n_{\pi_{2}}  \cdots n_{\pi_{r-1}} \quad \text{for} \ 1 < r \leq p.
\end{equation}
With $w_{\pi_1} = 1$, tensor elements of the $\pi_1$-th mode are contiguously stored in memory.
In contrast to hierarchical storage formats, all tensor elements with one differing multi-index element exhibit the same stride. 

%\subsection{Memory Access}
%Given an order-$p$ tensor with $\bar{n} = n_1 \cdot n_2 \cdots n_p$ contiguously stored elements, 
%Let $\mcJ = \{0,\dots,\bar{n}-1\}$ contain (relative) memory positions of an order-$p$ tensor with $\bar{n} = n_1 \cdot n_2 \cdots n_p$ contiguously stored elements.
%Let also $I_r = \{1,\dots,n_r\}$ be the $r$-th index set and $\mcI$ the Cartesian product of all index sets with $|\mcI| = |\mcJ|$.
%The relationship between elements of $\mcJ$ and $\mcI$ is defined in terms of the storage format or layout of a tensor. 
%Note that $|\mcI| = |\mcJ|$.
%, there must be a bijection between the elements of both sets for a given storage format.
%Let $\mbpi$ be a layout tuple and let $\mbw$ be its corresponding stride tuple according to Eq. \eqref{equ:stride.tuple}.
The location of tensor elements within the allocated memory space is determined by the storage format of a tensor and the corresponding \tit{layout} \tit{function}.
For a given layout and stride tuple, a layout function $\lambda_{\mbw} : \mcI \rightarrow \mcJ$ maps a multi-index to a scalar index according to 
\begin{equation}
\label{equ:lambda}
\lambda_{\mbw}(\mbi) = \sum_{r=1}^p w_r (i_r-1)
\end{equation}
%Using round brackets $\mubA(i_1,i_2,\dots,i_p)$ or $\mubA(\mbi)$ together with a multi-index $\mbi = (i_1,i_2,\dots,i_p)$ to identify tensor elements, 
With $j = \lambda_{\mbw} (\mbi)$ being the relative memory position of an element with a multi-index $\mbi$, reading from and writing to memory is accomplished with $j$ and the first element's address of $\mubA$.

%
%from \cite{bassoy:2018:fast,rogers:2016:efficient}
%\end{equation}
%the following mappings include any non-hierarchical layouts that can be specified by a layout tuple $\mbpi$.\todo{Fertigstellen.} 
%The layout function $\lambda_{\mbw}$ with 
%The inverse layout function $\lambda_{\mbw}^{-1}  : \mcJ \rightarrow \mcI$ of $\lambda_{\mbw}$ with $\mbi = \lambda_{\mbw}^{-1}(j)$ transforms scalar indices with\todo{This can go if the only shifting is sufficient.}
%\begin{equation}
%\label{equ:lambda_inv}
%i_{r} = \left\lfloor \frac{k_{r}}{w_{r}} \right\rfloor+1 
%\quad \text{with} \quad 
%k_{\pi_r} = k_{\pi_{r+1}} - w_{\pi_{r+1}} \cdot i_{\pi_{r+1}} \ \text{for} \ r < p
%\end{equation}
%and $i_{\pi_p} =  \lfloor j/ w_{\pi_p} \rfloor+1$. 


%Both layout functions are also valid for transforming multi-index and scalar indices for a subtensor.
%With  $k_r \in K_r$ we will use round brackets again $\mubA'(k_1,\dots,k_p)$ or $\mubA'(\mbk)$ to identify subtensor elements.
%The correspondence between tensor and subtensor elements for the $r$-th mode is given by the linear function $\gamma_r(k_r) = f_r + (k_r-1) \cdot t_r$ such that $\mubS(k_1,\dots,k_p) = \mubA(\gamma_1(k_1),\dots,\gamma_p(k_p))$.
%Subtensor elements can be accessed with single indices by applying the function $\lambda_{\mbw} \circ \gamma \circ \gamma_{\mbv}$ where $\mbv$ and $\mbw$ are stride tuples of a subtensor and tensor. 

%We can analogously define a scalar index set $\mcJ'$ for a subtensor with $\bar{n}'$ elements where $\bar{n}' = \prod_{r=1}^{p} n_r'$.
%Note that $\lambda$ can only be applied if $1 = f_r$, $l_r = n_r$ and $1 = t_r$ such that $n_r' = n_r$. 
%The layout function $\lambda$ however cannot be directly applied if any index triplet $(f_r,t_r,l_r)$ satisfies $1 < f_r$, $l_r < n_r$ or $1 < t_r$ such that $n_r' < n_r$. 
%such that $j = \lambda_{\mbw} \left( \gamma \left ( \lambda_{\mbw'}^{-1} \left (j' \right ) \right ) \right )$ 


\subsubsection{Tensor-Vector Multiplication}
Let $\mubA$ be an order-$p$ input tensor with a dimension tuple $\mbn = (n_1,\dots,n_q,\dots,n_p)$ and let $\mbb$ be a vector of length $n_q$ with $p>1$. 
Let $\mubC$ be a tensor with $p-1$ modes with a dimension tuple $\mbm =(n_1,\dots,n_{q-1},n_{q+1},\dots,n_p)$.
A mode-$q$ tensor-vector multiplication is denoted by $\mubC = \mubA \times_q \mbb$ where
\begin{equation}
\label{equ:tensor.vector.multplikation}
c_{i_1, \dots, i_{q-1}, i_{q+1}, \dots, i_p} = \sum_{i_q=1}^{n_q} a_{i_1, \dots, i_q, \dots, i_p} \cdot b_{i_q}
\end{equation}
is an element of $\mubC$.
Eq. \eqref{equ:tensor.vector.multplikation} is an inner product of a fiber of $\mubA$ and $\mbb$.% with $1 \leq i_r \leq n_r$ and $1 \leq r \leq p$.
The mode $q$ is its \tit{contraction} \tit{mode}.
We additionally term $\mbpi$ as the layout tuple of the input tensor $\mubA$ with a stride tuple $\mbw$ that is given by Eq. \eqref{equ:stride.tuple}.
With no transposition of $\mubA$ or $\mubC$, elements of the layout tuple $\mbvarphi$ of the mode-$q$ tensor-vector product $\mubC$ are given by
%Für die nachfolgende Untersuchung der Tensor"=Vektor"=Multiplikation ist zudem gefordert, dass die Operation keine Transponierung des Ein"= oder Ausgabetensors durchführen muss. 
%Ein Eingabetensor $\mubA$ der Stufe $p$ mit einem Anordungstupel $\mbpi$ der Länge $p$, soll der Ausgabetensor $\mubC$ der Stufe $p-1$ mit $\mubC = \mubA \bullet_m \mbb$ einen Anordnungstupel $\mbvarphi$ der Länge $p-1$ besitzen, dessen Werte sich von $\mbpi$ wie Folgt zusammensetzen.
\begin{equation}
\label{equ:tensor.times.vector.output.format}
\varphi_j = 
\begin{cases}
\pi_k       & \text{if} \ \pi_k < \pi_q\\ %  \wedge k < m 
\pi_k-1     & \text{if} \ \pi_k > \pi_q %  \wedge k < m 
\end{cases}
\quad \text{and} \quad
j = 
\begin{cases}
k   & \text{if} \ k < q \\
k-1 & \text{if} \ k > q 
\end{cases} 
\end{equation}
for $k=1,\dots,p$.
The stride tuple $\mbv$ is given by Eq. \eqref{equ:stride.tuple} using the shape $\mbm$ and permutation tuple $\mbvarphi$ of $\mubC$. 
%The tensor-vector multiplication is a generalization of the two-dimensional case from the computational perspective.
%It computes a vector-matrix product for $p=2$ and $q=1$ and a matrix-vector product for $p=2$ and $q=2$. 
%Its arithmetic intensity is equal to that of a matrix-vector multiplication.
%Categorized in \cite{dinapoli:2014:towards.efficient.use} as an operation of the tensor contraction class 2, its computation is likely to be limited by the memory bandwidth.


\begin{comment}
\subsection{Basic Linear Algebra Subroutines (BLAS)}
The BLAS specification provide efficient and portable subroutines with three levels. 
Subroutines herafter called routines of level three have the highest arithmetic intensity that can reach the peak performance of a computing system. 
In this work, 
\end{comment}
