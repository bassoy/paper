\section{Algorithm Design}
\label{sec:design}
\subsection{Baseline Algorithm with Contiguous Memory Access}
\label{sec:design:modified.baseline.algorithm}
The tensor-times-matrix multiplication in equation \ref{equ:tensor.matrix.multplication} can be implemented with one sequential algorithm using a nested recursion \cite{bassoy:2018:fast}.
It consists of two \ttt{if} statements with an \ttt{else} branch that computes a fiber-matrix product with two loops.
The outer loop iterates over the dimension $m$ of $\mubC$ and $\mbB$, while the inner iterates over dimension $n_q$ of $\mubA$ and $\mbB$ computing an inner product with fibers of $\mubA$ and $\mbB$. 
While matrix $\mbB$ can be accessed contiguously depending on its storage format, elements of $\mubA$ and $\mubC$ are accessed non-contiguously if $\pi_1 \neq q$.

%The access pattern can be improved by reordering tensor elements according to the storage format.
%However, copy operations reduce the overall throughput of the operation, see \cite{shi:2016:tensor.contraction}.


A better approach is illustrated in algorithm \ref{alg:ttm.sequential.coalesced} where the loop order is adjusted to the tensor layout $\mbpi$ and memory is accessed contiguously for $\pi_1 \neq q$ and $p > 1$.
The adjustment of the loop order is accomplished in line 5 which uses the layout tuple $\mbpi$ to select a multi-index element $i_{\pi_r}$ and to increment it with the corresponding stride $w_{\pi_r}$.
Hence, with increasing recursion level and decreasing $r$, indices are incremented with smaller strides as $w_{\pi_r} \leq w_{\pi_{r+1}}$.
%The condition of the second \ttt{if} statement in line 4 is changed from $r \geq 1$ to $r > 1$.
The second \ttt{if} statement in line number 4 allows the loop over mode $\pi_1$ to be placed into the base case which contains three loops performing a slice-matrix multiplication.
%the minimum stride $w_{\pi_1}$ to be included in the base case.
%The latter now contains three loops performing a slice-matrix multiplication. 
%The loop ordering are adjusted according to the tensor and matrix layout.
In this way, the inner-most loop is able to increment $i_{\pi_1}$ with a unit stride and contiguously accesses tensor elements of $\mubA$ and $\mubC$.
The second loop increments $i_q$ with which elements of $\mbB$ are contiguously accessed if $\mbB$ is stored in the row-major format.
The third loop increments $j$ and could be placed as the second loop if $\mbB$ is stored in the column-major format.
%The loop ordering are adjusted according to the tensor and matrix layout.
%The simple ordering of the three loops is discussed in \cite{golub:2013:matrix.computations}.

\begin{algorithm}[t]
%\SetAlgoNoLine
\DontPrintSemicolon
\SetKwProg{Fn}{}{}{end}
\SetKwFunction{function}{ttm}%
%\SetAlgoNoEnd
\footnotesize 
\SetAlgoVlined
\hrule
\BlankLine
\Fn{\function{$\mubA, \mbB, \mubC, \mbn, \mbpi, \mbi, m, q, \mhq, r$}}
{
	\uIf{$r = \mhq$ }
	{
		\function{$\mubA, \mbB, \mubC, \mbn, \mbpi, \mbi, m, q, \mhq, r-1$ }
	}
	\uElseIf{$r > 1$ }
	{
		\For{$i_{\pi_r} \leftarrow 1$ \KwTo $ n_{\pi_r}$}
		{
			\function{$\mubA, \mbB, \mubC, \mbn, \mbpi, \mbi, m, q, \mhq, r-1$}\;
		}		
	}	
	\Else%{$r\geq1 \wedge m \neq 1$}
	{
		\For{$j \leftarrow 1$ \KwTo $m$}
		{
			\For{$i_q \leftarrow 1$ \KwTo $n_q$}
			{			
				\For{$i_{\pi_1} \leftarrow 1$ \KwTo $n_{\pi_1}$}
				{
					$\mubC([\mbi_1,j,\mbi_2])$ \ttt{+=} $\mubA([\mbi_1,i_q,\mbi_2]) \cdot \mbB(j,i_q)$\;
				}
			}
		}
	}
}
\BlankLine
\hrule
\caption{
\footnotesize %
Modified baseline algorithm with contiguous memory access for the tensor-matrix multiplication.
The tensor order $p$ must be greater than $1$ and the contraction mode $q$ must satisfy $1 \leq q \leq p$ and $\pi_1 \neq q$.
The initial call must happen with $r=p$ where $\mbn$ is the shape tuple of $\mubA$ and $m$ is the $q$-th dimension of $\mubC$. 
%Iteration along mode $q$ with $\mhq = \mbpi^{-1}_q$ is moved into the inner-most recursion level.
\label{alg:ttm.sequential.coalesced}
}
\end{algorithm}

While spatial data locality is improved by adjusting the loop ordering, slices $\mubA_{\pi_1,q}'$, fibers $\mubC_{\pi_1}'$ and elements $\mubB(j,i_q)$ are accessed $m$, $n_q$ and $n_{\pi_1}$ times, respectively.
The specified fiber of $\mubC$ might fit into first or second level cache, slice elements of $\mubA$ are unlikely to fit in the local caches if the slice size $n_{\pi_1} \times n_q$ is large, leading to higher cache misses and suboptimal performance.
Instead of optimizing for better temporal data locality, we use existing high-performance BLAS implementations for the base case.
The following subsection explains this approach.

\subsection{BLAS-based Algorithms with Tensor Slices}
\label{sec:design:blas.based.algorithm.slices}
%\vspace{-0.3em}
\begin{table*}[t]
%\captionsetup{width=0.7\textheight}
\centering
\footnotesize
%\scriptsize
\begin{tabular}{ c c c c c c c c c c c c c c c } % 
\toprule
Case & Order $p$ & Layout $\mbpi_{\mubA,\mubC}$ & Layout $\mbpi_{\mbB}$ & Mode $q$ & Routine & \tf{T} & \tf{M} & \tf{N} & \tf{K} & \tf{A} & \tf{LDA} & \tf{B} & \tf{LDB} & \tf{LDC} \\
\midrule
1 & $1$ & -       & \tf{rm/cm} & $1$   & \tf{gemv} & -       & $m$   & $n_1$ & -     & $\mbB$  & $n_1$ & $\mubA$  & - & - \\
\midrule
2 & $2$ & \tf{cm} & \tf{rm} & $1$      & \tf{gemm} & $\mbB$  & $n_2$ & $m$   & $n_1$ & $\mubA$ & $n_1$ & $\mbB$   & $n_1$ & $m$   \\
  & $2$ & \tf{cm} & \tf{cm} & $1$      & \tf{gemm} & -       & $m$   & $n_2$ & $n_1$ & $\mbB$  & $m$   & $\mubA$  & $n_1$ & $m$   \\
3 & $2$ & \tf{cm} & \tf{rm} & $2$      & \tf{gemm} & -       & $m$   & $n_1$ & $n_2$ & $\mbB$  & $n_2$ & $\mubA$  & $n_1$ & $n_1$ \\
  & $2$ & \tf{cm} & \tf{cm} & $2$      & \tf{gemm} & $\mbB$  & $n_1$ & $m$   & $n_2$ & $\mubA$ & $n_1$ & $\mbB$   & $m$   & $n_1$ \\  
4 & $2$ & \tf{rm} & \tf{rm} & $1$      & \tf{gemm} & -       & $m$   & $n_2$ & $n_1$ & $\mbB$  & $n_1$ & $\mubA$  & $n_2$ & $n_2$ \\
  & $2$ & \tf{rm} & \tf{cm} & $1$      & \tf{gemm} & $\mbB$  & $n_2$ & $m$   & $n_1$ & $\mubA$ & $n_2$ & $\mbB$   & $m$   & $n_2$ \\
5 & $2$ & \tf{rm} & \tf{rm} & $2$      & \tf{gemm} & $\mbB$  & $n_1$ & $m$   & $n_2$ & $\mubA$ & $n_2$ & $\mbB$   & $n_2$ & $m$   \\
  & $2$ & \tf{rm} & \tf{cm} & $2$      & \tf{gemm} & -       & $m$   & $n_1$ & $n_2$ & $\mbB$  & $m$   & $\mubA$  & $n_2$ & $m$   \\
\midrule
6 & $>2$ & any    & \tf{rm} & $\pi_1$  & \tf{gemm} & $\mbB$  & $\mbnq$ & $m$     & $n_q$ & $\mubA$ & $n_q$   & $\mbB$  & $n_q$   & $m$\\
  & $>2$ & any    & \tf{cm} & $\pi_1$  & \tf{gemm} & -       & $m$     & $\mbnq$ & $n_q$ & $\mbB$  & $m$     & $\mubA$ & $n_q$   & $m$\\
7 & $>2$ & any    & \tf{rm} & $\pi_p$  & \tf{gemm} & -       & $m$     & $\mbnq$ & $n_q$ & $\mbB$  & $n_q$   & $\mubA$ & $\mbnq$ & $\mbnq$ \\
  & $>2$ & any    & \tf{cm} & $\pi_p$  & \tf{gemm} & $\mbB$  & $\mbnq$ & $m$     & $n_q$ & $\mubA$ & $\mbnq$ & $\mbB$  & $m$     & $\mbnq$ \\
\midrule
8 & $>2$ & any    & \tf{rm} & $\pi_2,..,\pi_{p-1}$ & \tf{gemm*} & -      & $m$ & $n_{\pi_1}$ & $n_q$ & $\mbB$  & $n_q$ & $\mubA$ & $w_q$ & $w_q$ \\
  & $>2$ & any    & \tf{cm} & $\pi_2,..,\pi_{p-1}$ & \tf{gemm*} & $\mbB$ & $n_{\pi_1}$ & $m$ & $n_q$ & $\mubA$ & $w_q$ & $\mbB$  & $m$   & $w_q$ \\
\bottomrule
\end{tabular}
%\vspace{0.2cm}
\caption%
{%
\footnotesize
Eight cases of CBLAS functions \tf{gemm} and \tf{gemv} implementing the mode-$q$ tensor-matrix multiplication with a row-major or column-major format.
%Argument for \tf{gemv} and \tf{gemm} with 
Arguments \tf{T}, \tf{M}, \tf{N}, etc. of \tf{gemv} and \tf{gemm} are chosen with respect to the tensor order $p$, layout $\mbpi$ of $\mubA$, $\mbB$, $\mubC$ and contraction mode $q$ where \tf{T} specifies if $\mbB$ is transposed.
Function \tf{gemm*} with a star denotes multiple \tf{gemm} calls with different tensor slices.
Argument $\bar{n}_q$ for case 6 and 7 is defined as $\bar{n}_q = (\prod_r^p n_r)/n_q$.
Input matrix $\mbB$ is either stored in the column-major or row-major format.
The storage format flag set for \tf{gemm} and \tf{gemv} is determined by the element ordering of $\mbB$.
}
\label{tab:mapping_rm_cm}
\end{table*}
Algorithm \ref{alg:ttm.sequential.coalesced} computes the mode-$q$ tensor-matrix product in a recursive fashion.
The base case multiplies different tensor slices of $\mubA$ with the matrix $\mbB$.
Instead of optimizing the slice-matrix multiplication in the base case, one can use a \ttt{gemm} routine instead.
Note that algorithm \ref{alg:ttm.sequential.coalesced} is the general case for $p\geq 2$ but only executable for $\pi_1 \neq q$.
For $\pi_1 = q$, the tensor-matrix product can be computed by a matrix-matrix multiplication where the input tensor $\mubA$ can be flattened into a matrix without any copy operation.
The same can be applied when $\pi_p = q$ and five other cases where the input tensor is either one or two-dimensional.
In summary, there are seven other corner cases to the general case where a single \ttt{gemv} or \ttt{gemm} call suffices to compute the tensor-matrix product.
All eight cases per storage format are listed in table \ref{tab:mapping_rm_cm}.
The arguments of the CBLAS routines \ttt{gemv} or \ttt{gemm} are set according to the tensor order $p$, tensor layout $\mbpi$ and contraction mode $q$.
If the input matrix $\mbB$ has the row-major order, parameter \ttt{CBLAS\_ORDER} of function \ttt{gemm} is set to \ttt{CblasRowMajor} and \ttt{CblasColMajor} otherwise.
Note that table \ref{tab:mapping_rm_cm} supports all linear tensor layouts of $\mubA$ and $\mubC$ with no limitations on tensor order and contraction mode.
The following subsection describes all eight cases when the input matrix $\mbB$ has the row-major order.

%Note , all linear tensor layouts can be supported by setting the remaining parameters of \ttt{gemm}.

%TODO: previously explain how CBLAS is called and what options one has, i.e. explain the design space
%TODO: explain the table for B format (already explained for RM) and CBLAS multiplication (already explained for RM) as on subsubsection and include two other subsection explaining what changes if we take B.format CM and another subsection explaining how the parameters change if it is CM.

%We apply highly optimized routines to fully or partly execute tensor contractions as it is done in \cite{li:2015:input, shi:2016:tensor.contraction}.
%The function and parameter configurations for the tensor multiplication can be divided into eight cases.

%Table \ref{tab:mapping} extends the finding in \cite{dinapoli:2014:towards.efficient.use} precisely defining the mapping for any storage format. 
%It also complies with the findings in \cite{li:2015:input}.
\subsubsection{Row-Major Matrix Multiplication}

\tit{Case 1:}
If $p=1$, The tensor-vector product $\mubA \times_1 \mbB$ can be computed with a \ttt{gemv} operation where $\mubA$ is an order-$1$ tensor $\mba$ of length $n_1$ such that $\mba^T \cdot \mbB$.

\tit{Case 2-5:}
If $p=2$, $\mubA$ and $\mubC$ are order-$2$ tensors with dimensions $n_1$ and $n_2$.
In this case the tensor-matrix product can be computed with a single \ttt{gemm}.
If $\mbA$ and $\mbC$ have the column-major format with $\mbpi=(1,2)$, \ttt{gemm} either executes $\mbC = \mbA \cdot \mbB^T$ for $q =1$ or $\mbC = \mbB \cdot \mbA$ for $q=2$.
Both matrices can be interpreted $\mbC$ and $\mbA$ as matrices in row-major format although both are stored column-wise.
If $\mbA$ and $\mbC$ have the row-major format with $\mbpi=(2,1)$, \ttt{gemm} either executes $\mbC = \mbB \cdot \mbA$ for $q =1$ or $\mbC = \mbA \cdot \mbB^T$ for $q=2$. 
The transposition of $\mbB$ is necessary for the cases 2 and 5 which is independent of the chosen layout.

\tit{Case 6-7 :}
% If the order of $\mubA$ and $\mubC$ is greater than $2$ 
%the contraction mode $q$ is equal to $\pi_1$ 
If $p>2$ and if $q=\pi_1$(case 6), a single \ttt{gemm} with the corresponding arguments executes $\mbC = \mbA \cdot \mbB^T$ and computes a tensor-matrix product $\mubC = \mubA \times_{\pi_1} \mbB$.
% for any storage layout of $\mubA$ and $\mubC$
Tensors $\mubA$ and $\mubC$ are flattened with $\varphi_{2,p}$ to row-major matrices $\mbA$ and $\mbC$.
%$f_{2,p}$, see subsection \ref{sec:preliminaries:flattening}.
Matrix $\mbA$ has $\bar{n}_{\pi_1} = \bar{n} / n_{\pi_1}$ rows and $n_{\pi_1}$ columns while matrix $\mbC$ has the same number of rows and $m$ columns.
If $\pi_p=q$ (case 7), $\mubA$ and $\mubC$ are flattened with $\varphi_{1,p-1}$ to column-major matrices $\mbA$ and $\mbC$.
Matrix $\mbA$ has $n_{\pi_p}$ rows and $\bar{n}_{\pi_p} =  \bar{n} / n_{\pi_p}$ columns while $\mbC$ has $m$ rows and the same number of columns.
In this case, a single \ttt{gemm} executes $\mbC = \mbB \cdot \mbA$ and computes $\mubC = \mubA \times_{\pi_p} \mbB$.
Noticeably, the desired contraction are performed without copy operations, see subsection \ref{sec:preliminaries:flattening.reshaping}. 

\tit{Case 8 $(p>2)$:}
If the tensor order is greater than $2$ with $\pi_1\neq q$ and $\pi_p \neq q$, the modified baseline algorithm \ref{alg:ttm.sequential.coalesced} is used to successively call $\bar{n} / (n_q \cdot n_{\pi_1})$ times \ttt{gemm} with different tensor slices of $\mubC$ and $\mubA$.
Each \ttt{gemm} computes one slice $\mubC_{\pi_1,q}'$ of the tensor-matrix product $\mubC$ using the corresponding tensor slices $\mubA_{\pi_1,q}'$ and the matrix $\mbB$.
The matrix-matrix product $\mbC = \mbB \cdot \mbA$ is performed by interpreting both tensor slices as row-major matrices $\mbA$ and $\mbC$ which have the dimensions $(n_q,n_{\pi_1})$ and $(m,n_{\pi_1})$, respectively.

\subsubsection{Column-Major Matrix Multiplication}
The tensor-matrix multiplication is performed with the column-major version of \ttt{gemm} when the input matrix $\mbB$ is stored in column-major order.
Although the number of \ttt{gemm} cases remains the same, the \ttt{gemm} arguments must be rearranged.
The argument arrangement for the column-major version can be derived from the row-major version that is provided in table \ref{tab:mapping_rm_cm}.

Firstly, the BLAS arguments of \ttt{M} and \ttt{N}, as well as \ttt{A} and \ttt{B} must be swapped.
Additionally, the transposition flag for matrix $\mbB$ is toggled.
Also, the leading dimension argument of \ttt{A} is swapped to \ttt{LDB} or \ttt{LDA}.
The only new argument is the new leading dimension of \ttt{B}.

Given case 4 with the row-major matrix multiplication in table \ref{tab:mapping_rm_cm} where tensor $\mubA$ and matrix $\mbB$ are passed to \ttt{B} and \ttt{A}.
The corresponding column-major version is attained when tensor $\mubA$ and matrix $\mbB$ are passed to \ttt{A} and \ttt{B} where the transpose flag for $\mbB$ is set and the remaining dimensions are adjusted accordingly.


\subsubsection{Matrix Multiplication Variations}
The column-major and row-major versions of \ttt{gemm} can be used interchangeably by adapting the storage format. 
This means that a \ttt{gemm} operation for column-major matrices can compute the same matrix product as one for row-major matrices, provided that the arguments are rearranged accordingly.
While the argument rearrangement is similar, the arguments associated with the matrices \ttt{A} and \ttt{B} must be interchanged.
Specifically, \ttt{LDA} and \ttt{LDB} as well as \ttt{M} and \ttt{N} are swapped along with the corresponding matrix pointers.
In addition, the transposition flag must be set for \ttt{A} or \ttt{B} in the new format if \ttt{B} or \ttt{A} is transposed in the original version.

For instance, the column-major matrix multiplication in case 4 of table \ref{tab:mapping_rm_cm} requires the arguments of \ttt{A} and \ttt{B} to be tensor $\mubA$ and matrix $\mbB$ with $\mbB$ being transposed.
The arguments of an equivalent row-major multiplication for \ttt{A}, \ttt{B}, \ttt{M}, \ttt{N}, \ttt{LDA}, \ttt{LDB} and \ttt{T} are then initialized with $\mbB$, $\mubA$, $m$, $n_2$, $m$, $n_2$ and $\mbB$.

Another possible matrix multiplication variant with the same product is computed when, instead of $\mbB$, tensors $\mubA$ and $\mubC$ with adjusted arguments are transposed.
We assume that such reformulations of the matrix multiplication do not outperform the variants shown in Table \ref{tab:mapping_rm_cm}, as we expect highly optimized BLAS libraries to adjust .


\subsection{Matrix Multiplication with Subtensors}
\label{sec:design:blas.based.algorithm.subtensors}
Algorithm \ref{alg:ttm.sequential.coalesced} can be slightly modified in order to call \ttt{gemm} with flattened order-$\mhq$ subtensors that correspond to larger tensor slices.
Given the contraction mode $q$ with $1 < q < p$, the maximum number of additionally fusible modes is $\mhq-1$ with $\mhq = \mbpi^{-1}(q)$ where $\mbpi^{-1}$ is the inverse layout tuple.
The corresponding fusible modes are therefore $\pi_1,\pi_2,\dots,\pi_{\mhq-1}$.

The non-base case of the modified algorithm only iterates over dimensions that have indices larger than $\mhq$ and thus omitting the first $\mhq$ modes.
The conditions in line 2 and 4 are changed to $1 < r \leq \mhq$ and $\mhq < r$, respectively.
Thus, loop indices belonging to the outer $\pi_r$-th loop with $\mhq+1 \leq r \leq p$ define the order-$\mhq$ subtensors $\mubA_{\mbpi'}'$ and $\mubC_{\mbpi'}'$ of $\mubA$ and $\mubC$ with $\mbpi' = (\pi_1,\dots,\pi_{\mhq-1},q)$.
Flattening the subtensors $\mubA_{\mbpi'}'$ and $\mubC_{\mbpi'}'$ with $\varphi_{1,\mhq-1}$ for the modes $\pi_1,\dots,\pi_{\mhq-1}$ yields two tensor slices with dimension $n_q$ or $m$ and the fused dimension $\bar{n}_q = \prod_{r=1}^{\mhq-1} n_{\pi_r}$ with $\bar{n}_q = w_q$.
Both tensor slices can be interpreted either as row-major or column-major matrices with shapes $(n_q,\bar{n}_q)$ or $(w_q,\bar{n}_q)$ in case of $\mubA$ and $(m,\bar{n}_q)$ or $(\bar{n}_q,m)$ in case of $\mubC$, respectively.

The \ttt{gemm} function in the base case is called with almost identical arguments except for the parameter $M$ or $N$ which is set to $\bar{n}_q$ for a column-major or row-major multiplication, respectively.
Note that neither the selection of the subtensor nor the flattening operation copy tensor elements.
This description supports all linear tensor layouts and generalizes lemma 4.2 in \cite{li:2015:input} without copying tensor elements, see section \ref{sec:preliminaries:flattening.reshaping}.

%\tf{rm} & 
% -      & $m$ & $n_{\pi_1}$ & $n_q$ & $\mbB$  & $n_q$ & $\mubA$ & $w_q$ & $w_q$ \\

%\tf{cm}
% $\mbB$ & $n_{\pi_1}$ & $m$ & $n_q$ & $\mubA$ & $w_q$ & $\mbB$  & $m$   & $w_q$ \\


% It is not possible to write out the new layout and shape tuples as the layout elements are adjusted depending on the layout elements ($s_k$) and the layout element index ($j_k$).

 
\subsection{Parallel BLAS-based Algorithms}
\label{subsec:parallel.multi-loops}
Most BLAS libraries provide API functions for adjusting the number of threads.
Hence, functions such as \ttt{gemm} and \ttt{gemv} can be run either using a single or multiple threads.
For the cases one to seven with a single BLAS call, the number of threads is always set to the number of available cores.
The following subsections discuss parallel versions for the eighth case in which the outer loops of algorithm \ref{alg:ttm.sequential.coalesced} and the \ttt{gemm} function inside the base case can be run in parallel.
Note that the parallelization strategies can be combined with the aforementioned slicing methods.

\subsubsection{Sequential Loops and Parallel Matrix Multiplication}
Similar to the first seven cases, algorithm \ref{alg:ttm.sequential.coalesced} does not need to be modified except for enabling \ttt{gemm} to run multi-threaded in the base case by using BLAS function calls to adjust the thread count.
This type of parallelization strategy might be beneficial with order-$\mhq$ tensor slices where the contraction mode satisfies $q = \pi_{p-1}$, the inner dimensions $n_{\pi_1},\dots,n_{\mhq}$ are large and the outer-most dimension $n_{\pi_{p}}$ is smaller than the available processor cores.
For instance, given a first-order storage format and the contraction mode $q$ with $q=p-1$ and $n_p=2$, the dimensions of flattened order-$q$ tensor slices are $\prod_{r=1}^{p-2}n_r$ and $n_{p-1}$.
This allows \ttt{gemm} to be executed with large dimensions using multiple threads increasing the likelihood to reach a high throughput.
However, if the above conditions are not met, a multi-threaded \ttt{gemm} operates on small tensor slices which might lead to an suboptimal utilization of the available cores.
This algorithm version will be referred to as \ttt{<par-gemm>}.
Depending on the subtensor shape, we will either add \ttt{<slice>} for order-$2$ subtensors or \ttt{<subtensor>} for order-$\mhq$ subtensors with $\mhq = \pi^{-1}_q$.

\begin{algorithm}[t]
\footnotesize
\DontPrintSemicolon
\SetAlgoVlined
\SetKwProg{Fn}{}{}{end}
\SetKwFunction{Func}{ttm<par-loop><slice>}%
\SetKwFunction{GEMM}{gemm}%
\SetKwFunction{flatten}{flatten}%
\SetKw{KwWith}{with}
\SetKw{KwStep}{step}
\SetKw{KwThreads}{threads}
\SetKwFor{ParallelFor}{parallel for}{do}{endfor}
\hrule
\BlankLine
%\Fn{\Func{$\mubA'$, $\mubA'$$\mubC'$, $$ $\mbn'$, $\mbw'$, $m$}}
\Fn{\Func{$\mubA$, $\mbB$, $\mubC$, $\mbn$, $\mbpi$, $m$, $q$, $p$}}
{
	%[$\mubA'$,$\mbn'$,$\mbw'$] $=$ \flatten($\mubA$, $\mbn$, $\mbpi$, $q$, $p$)\;
	%[$\mubC'$,\_,\_] $=$ \flatten($\mubC$, $\mbm$, $\mbpi$, $q$, $p$)\;
	%[$\mbn'$,$\mbw'$] $=$ \flatten($\mbn$, $\mbpi$, $q$, $p$)\;
	[$\mubA'$, $\mubC'$, $\mbn'$, $\mbw'$] $=$ \flatten($\mubA$, $\mubC$, $\mbn$, $m$, $\mbpi$, $q$, $p$)\;
	\ParallelFor{$i \leftarrow 1$ \KwTo $n_{4}'$}
	{
		\ParallelFor{$j \leftarrow 1$ \KwTo $n_{2}'$}
		{
			\GEMM{$m$, $n_{1}'$, $n_{3}'$, $1$, $\mbB$,$n_{3}'$, $\mubA_{ij}'$,$w_{3}'$, $0$, $\mubC_{ij}'$,$w_{3}'$}\;
		}
	}
}
\BlankLine
\hrule
\caption{%
\footnotesize
Function \tf{ttm<par-loop><slice>} is an optimized version of Algorithm \ref{alg:ttm.sequential.coalesced}.
The \tf{flatten} function transforms the order-$p$ tensors $\mubA$ and $\mubC$ with layout tuple $\mbpi$ and their respective dimension tuples $\mbn$ and $\mbm$ into order-$4$ tensors $\mubA'$ and $\mubC'$ with layout tuple $\mbpi'$ and their respective dimension tuples $\mbn'$ and $\mbm'$ where $\mbn' = (n_{\pi_1},\mhn_{\pi_2},n_q,\mhn_{\pi_4})$ and $m_3' = m$ and $n_k' = m_k'$ for $k \neq 3$.
Each thread calls multiple single-threaded \tf{gemm} functions each of which executes a slice-matrix multiplication with the order-$2$ tensor slices $\mubA_{ij}'$ and $\mubC_{ij}'$.
Matrix $\mbB$ has the row-major storage format.
\label{alg:ttm.slice.fused.parallel}
}
\end{algorithm}


\subsubsection{Parallel Loops and Sequential Matrix Multiplication}
Instead of sequentially calling multi-threaded \ttt{gemm}, it is also possible to call single-threaded \ttt{gemm}s in parallel.
Similar to the previous approach, the matrix multiplication can be performed with tensor slices or order-$\mhq$ subtensors.

\paragraph{Matrix Multiplication with Tensor Slices}
Algorithm \ref{alg:ttm.slice.fused.parallel} with function \ttt{ttm}\ttt{<par-loop>}\allowbreak\ttt{<slice>} executes a single-threaded \ttt{gemm} with tensor slices in parallel using all modes except $\pi_1$ and $\pi_{\mhq}$.
The first statement of the algorithm calls the \ttt{flatten} function which transforms tensors $\mubA$ and $\mubC$ without copying elements by calling the flattening operation $\varphi_{\pi_{\mhq+1},\pi_p}$ and $\varphi_{\pi_{2},\pi_{\mhq-1}}$.
The resulting tensors $\mubA'$ and $\mubC'$ are of order $4$.
Tensor $\mubA'$ has the shape $\mbn'=(n_{\pi_1},\mhn_{\pi_2},n_{q}, \mhn_{\pi_4})$ with the dimensions $\mhn_{\pi_2} = \prod_{r=2}^{\mhq-1} n_{\pi_r}$ and $\mhn_{\pi_4} = \prod_{r=\mhq+1}^{p} n_{\pi_r}$.
Tensor $\mubC'$ has the same shape as $\mubA'$ with dimensions $m_r' = n_r'$ except for the third dimension which is given by $m_3=m$.

The following two \ttt{parallel for} loop constructs index all free modes.
The outer loop iterates over $n_4' = \mhn_{\pi_4}$ while the inner one loops over $n_2' = \mhn_{\pi_2}$ calling \ttt{gemm} with tensor slices $\mubA_{2,4}'$ and $\mubC_{2,4}'$.
Here, we assume that matrix $\mbB$ has the row-major format which is why both tensor slices are also treated as row-major matrices.
Notice that \ttt{gemm} in Algorithm \ref{alg:ttm.slice.fused.parallel} will be called with exact same arguments as displayed in the eighth case in table \ref{tab:mapping_rm_cm} where $n_{1}' = n_{\pi_1}$, $n_{3}' = n_q$ and $w_q = w_3'$. %$w_q = \prod_{r=1}^{\mhq-1} n_{\pi_r} = w_3' = n_{\pi_1} \cdot \mhn_{\pi_2} = \prod_{r=1}^{\mhq-1} n_{\pi_r}$
For the sake of simplicity, we omitted the first three arguments of \ttt{gemm} which are set to \ttt{CblasRowMajor} and \ttt{CblasNoTrans} for \ttt{A} and \ttt{B}.
With the help of the flattening operation, the tree-recursion has been transformed into two loops which iterate over all free indices.



\paragraph{Matrix Multiplication with Subtensors}
The following algorithm and the flattening of subtensors is a combination of the previous paragraph and subsection \ref{sec:design:blas.based.algorithm.subtensors}.
With order-$\mhq$ subtensors, only the outer modes $\pi_{\mhq+1},\dots,\pi_{p}$ are free for parallel execution while the inner modes $\pi_{1},\dots,\pi_{\mhq-1},q$ are used for the slice-matrix multiplication.
Therefore, both tensors are flattened twice using the flattening operations $\varphi_{\pi_{1},\pi_{\mhq-1}}$ and $\varphi_{\pi_{\mhq+1},\pi_p}$. 
Note that in contrast to tensor slices, the first flattening also contains the dimension $n_{\pi_{1}}$.
The flattened tensors are of order $3$ where $\mubA'$ has the shape $\mbn'=(\mhn_{\pi_1},n_{q},\mhn_{\pi_3})$ with $\mhn_{\pi_1} = \prod_{r=1}^{\mhq-1} n_{\pi_r}$ and $\mhn_{\pi_3} = \prod_{r=\mhq+1}^{p} n_{\pi_r}$.
Tensor $\mubC'$ has the same dimensions as $\mubA'$ except for $m_2=m$.

Algorithm \ref{alg:ttm.slice.fused.parallel} needs a minor modification  for supporting order-$\mhq$ subtensors.
Instead of two loops, the modified algorithm consists of a single loop which iterates over dimension $\mhn_{\pi_3}$ calling a single-threaded \ttt{gemm} with large tensor slices $\mubA'$ and $\mubC'$.
The shape and strides of both tensor slices as well as the function arguments of \ttt{gemm} have already been provided by the previous subsection \ref{sec:design:blas.based.algorithm.subtensors}.
This \ttt{ttm} version will referred to as \ttt{<par-loop>}\allowbreak\ttt{<subtensor>}.

Note that function with the tags \ttt{<par-gemm>} and \ttt{<par-loop>} implement opposing versions of the \ttt{ttm} where either \ttt{gemm} or the fused loop is performed in parallel.
Version \ttt{<par-loop-gemm} executes available loops in parallel where each loop thread executes a multi-threaded \ttt{gemm} with either subtensors or tensor slices.
%TODO: do we need the latter version to mention?

\subsubsection{Multithreaded Batched Matrix Multiplication}
The next version of the base algorithm is a modified version of the general subtensor-matrix approach that calls a single batched \ttt{gemm} for the eighth case.
The subtensor dimensions and remaining \ttt{gemm} arguments remain the same.
The library implementation is responsible how subtensor-matrix multiplications are executed and if subtensors are further divided into smaller subtensors or tensor slices.
This version will be referred to as the \ttt{<gemm\_batch>} variant.


\subsubsection{OpenMP Parallelization}
The two \ttt{parallel for} loops have been parallelized using the OpenMP directive \ttt{omp parallel for} together with the \ttt{schedule(static)}, \ttt{num\_threads(ncores)} and \ttt{proc\_bind}\allowbreak\ttt{(spread)} clauses.
In case of tensor-slices, the \ttt{collapse(2)} clause is added for transforming both loops into one loop which has an iteration space of the first loop times the second one. 

The \ttt{num\_threads(ncores)} clause specifies the number of threads within a team where \ttt{ncores} is equal to the number of processor cores. 
Hence, each OpenMP thread is responsible for computing $\bar{n}' / \text{\ttt{ncores}}$ independent slice-matrix products where $\bar{n}' = n_2' \cdot n_4'$ for tensor slices and $\bar{n}' = n_4'$ for mode-$\mhq$ subtensors.

The \ttt{schedule(static)} instructs the OpenMP runtime to divide the iteration space into almost equally sized chunks.
Each thread sequentially computes $ \bar{n}' / \text{\ttt{ncores}}$ slice-matrix products.
We decided to use this scheduling kind as all slice-matrix multiplications have the same number of floating-point operations with a regular workload where one can assume negligible load imbalance.
Moreover, we wanted to prevent scheduling overheads for small slice-matrix products were data locality can be an important factor for achieving higher throughput.

We did not set the \ttt{OMP\_PLACES} environment variable which defaults to the OpenMP \ttt{cores} setting defining a place as a single processor core. % 1 place <-> 1 core
Together with the clause \ttt{num\_threads(ncores)}, the number of OpenMP threads is equal to the number of OpenMP places, i.e. to the number of processor cores.
We did not measure any performance improvements for a higher thread count.

The \ttt{proc\_bind(spread)} clause additionally binds each OpenMP thread to one OpenMP place which lowers inter-node or inter-socket communication and improves local memory access.
Moreover, with the \ttt{spread} thread affinity policy, consecutive OpenMP threads are spread across OpenMP places which can be beneficial if the user decides to set \ttt{hwthreads} smaller than the number of processor cores.