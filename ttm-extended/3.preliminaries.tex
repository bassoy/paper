\section{Background}
\label{sec:preliminaries}

\subsection{Tensor Notation}
\label{sec:preliminaries:notation}
An order-$p$ tensor is a $p$-dimensional array  where tensor elements are contiguously stored in memory\cite{lim:2017:hypermatrices, lee:2018:fundamental}.
We write $a$, $\mba$, $\mbA$ and $\mubA$ in order to denote scalars, vectors, matrices and tensors. 
If not otherwise mentioned, we assume $\mubA$ to have order $p>2$.
The $p$-tuple $\mbn = (n_1,n_2,\dots,n_p)$ will be referred to as the shape or dimension tuple of a tensor where $n_r>1$.
We will use round brackets $\mubA(i_1,i_2,\dots,i_p)$ or $\mubA(\mbi)$ to denote a tensor element where $\mbi = (i_1,i_2,\dots,i_p)$ is a multi-index.
For convenience, we will also use square brackets to concatenate index tuples such that 
$[\mbi,\mbj] = (i_1,i_2,\dots,i_r,\allowbreak j_1,j_2,\dots,j_q)$ where $\mbi$ and $\mbj$ are multi-indices of length $r$ and $q$, respectively.

\subsection{Tensor-Matrix Multiplication}
Let $\mubA$ and $\mubC$ be order-$p$ tensors with shapes $\mbn_a = ([\mbn_1,n_q,\mbn_2])$ and $\mbn_c =([\mbn_1,m,\mbn_2])$ where $\mbn_1 = (n_1,n_2,\allowbreak\dots,n_{q-1})$ and $\mbn_2 = (n_{q+1},n_{q+2},\allowbreak\dots,n_p)$.
Let $\mbB$ be a matrix of shape $\mbn_b = (m,n_q)$.
A $q$-mode tensor-matrix product is denoted by $\mubC = \mubA \times_q \mbB$. 
An element of $\mubC$ is defined by
\begin{equation}
	\label{equ:tensor.matrix.multplication}
	\mubC([\mbi_1, j, \mbi_2]) = \sum_{i_q=1}^{n_q} \mubA([\mbi_1, i_q, \mbi_2]) \cdot \mbB(j,i_q)
\end{equation}
with $\mbi_1 = (i_1,\dots,i_{q-1})$, $\mbi_2 = (i_{q+1},\dots,i_p)$ where $1 \leq i_r \leq n_r$ and $1 \leq j \leq m$ \cite{li:2015:input, kolda:2009:decompositions}.
%Similar to the tensor-vector multiplication, the multiplication consists of multiple inner productd of a fiber of $\mubA$ and $\mbb$.% with $1 \leq i_r \leq n_r$ and $1 \leq r \leq p$.
Mode $q$ is called the contraction mode with $1 \leq q \leq p$.
The tensor-matrix multiplication generalizes the computational aspect of the two-dimensional case $\mbC = \mbB \cdot \mbA$ if $p=2$ and $q=1$.
Its arithmetic intensity is equal to that of a matrix-matrix multiplication and is not memory-bound.
%TODO: %Categorized in \cite{dinapoli:2014:towards.efficient.use} as an operation of the tensor contraction class 2, its computation is likely to be limited by the memory bandwidth.

In the following, we assume that the tensors $\mubA$ and $\mubC$ have the same tensor layout $\mbpi$. 
Elements of matrix $\mubB$ can be stored either in the column-major or row-major format.
The tensor-matrix multiplication with $i_q$ iterating over the second mode of $\mbB$ is also referred to as the $q$-mode product which is a building block for tensor methods such as the higher-order orthogonal iteration or the higher-order singular value decomposition \cite{kolda:2009:decompositions}.
Please note that the following method can be applied, if indices $j$ and $i_q$ of matrix $\mbB$ are swapped.

\subsection{Subtensors}
A subtensor references elements of a tensor $\mubA$ and is denoted by $\mubA'$.
It is specified by a selection grid that consists of $p$ index ranges.
In this work, an index range of a given mode $r$ shall either contain all indices of the mode $r$ or a single index $i_r$ of that mode where $1 \leq r \leq p$. 
Subtensor dimensions $n_r'$ are either $n_r$ if the full index range or $1$ if a a single index for mode $r$ is used.
Subtensors are annotated by their non-unit modes such as $\mubA_{u,v,w}'$ where $n_u > 1$, $n_v > 1$ and $n_w >1$ for $1 \leq u \neq v \neq w \leq p$.
The remaining single indices of a selection grid can be inferred by the loop induction variables of an algorithm. 
The number of non-unit modes determine the order $p'$ of subtensor where $1 \leq p' < p$.
In the above example, the subtensor $\mubA_{u,v,w}'$ has three non-unit modes and is thus of order $3$.
For convenience, we might also use an dimension tuple $\mbm$ of length $p'$ with $\mbm = (m_1,m_2,\dots,m_{p'})$ to specify a mode-$p'$ subtensor $\mubA_{\mbm}'$.
An order-$2$ subtensor of $\mubA'$ is a tensor slice $\mbA_{u,v}'$ and an order-$1$ subtensor of $\mubA'$ is a fiber $\mba_u'$.
%TODO: sollte man das doch nicht lieber umkehren um das eindeutig zu machen?

\subsection{Linear Tensor Layouts}
\label{sec:preliminaries:layout}
We use a layout tuple $\mbpi \in \bbN^p$ to encode all linear tensor layouts including the first-order or last-order layout.
They contain permuted tensor modes whose priority is given by their index.
For instance, the general $k$-order tensor layout for an order-$p$ tensor is given by the layout tuple $\mbpi$ with $\pi_r = k-r+1$ for $1 < r \leq k$ and $r$ for $k < r \leq p$.
The first- and last-order storage formats are given by $\mbpi_F = (1,2,\dots,p)$ and $\mbpi_{L} = (p,p-1,\dots,1)$.
An inverse layout tuple $\mbpi^{-1}$ is defined by $\mbpi^{-1}(\mbpi(k)) = k$.
Given a layout tuple $\mbpi$ with $p$ modes, the $\pi_r$-th element of a stride tuple is given by $w_{\pi_r} = \prod_{k=1}^{r-1} n_{\pi_k}$ for $1 < r \leq p$ and $w_{\pi_1} = 1$.
Tensor elements of the $\pi_1$-th mode are contiguously stored in memory.
The location of tensor elements is determined by the tensor layout and the layout function.
For a given tensor layout and stride tuple, a layout function $\lambda_{\mbw}$ maps a multi-index to a scalar index with $\lambda_{\mbw}(\mbi) = \sum_{r=1}^p w_r (i_r-1)$, see \cite{bassoy:2018:fast,pawlowski:2019:morton.tensor.computations}.
%Index $j = \lambda_{\mbw} (\mbi)$ is called the relative memory position of an element.
%Reading from and writing to memory is accomplished with $j$ and the first element's address of $\mubA$.
%\vspace{-1em}

\subsection{Flattening and Reshaping}
\label{sec:preliminaries:flattening.reshaping}
The following two operations define non-modifying reformatting transformations of dense tensors with contiguously stored elements and linear tensor layouts.

%\subsubsection{Flattening}
The flattening operation $\varphi_{u,v}$ transforms an order-$p$ tensor $\mubA$ with a shape $\mbn$ and layout $\mbpi$ tuple to an order-$p'$ view $\mubB$ with a shape $\mbm$ and layout $\mbtau$ tuple of length $p'$ with $p' = p-v+u$ and $1 \leq u < v \leq p$.
It is akin to tensor unfolding, also known as matricization and vectorization \cite[p.459]{kolda:2009:decompositions}.
However, it neither modifies the element ordering nor copies tensor elements.
Given a layout tuple $\mbpi$ of $\mubA$, the flattening operation $\varphi_{u,v}$ is defined for contiguous modes $\mbhpi = (\pi_u,\pi_{u+1}, \dots, \pi_{v})$ of $\mbpi$.
With $j_k = 0$ if $k \leq u$ and $j_k = v-u$ if $k>u$ where $1 \leq k \leq p'$, the resulting layout tuple $\mbtau = (\tau_1,\dots,\tau_{p'})$ of $\mubB$ is then given by $\tau_u = \min(\mbpi_{u,v})$ and
$\tau_k = \pi_{k+j_k} - s_k \quad \text{for} \ k \neq u$
with $s_k = \left| \{ \pi_i \mid \pi_{k+j_k} > \pi_i \wedge \pi_i \neq \min(\mbhpi) \wedge u \leq i \leq p \} \right|$.
Elements of the shape tuple $\mbm$ are defined by $m_{\tau_u} = \prod_{k=u}^v n_{\pi_k}$ and $m_{\tau_k} = n_{\pi_{k+j}}$ for $k \neq u$.

%\subsubsection{Reshaping}
%Reshaping $\rho$ transforms an order-$p$ tensor $\mubA$ to another order-$p$ tensor $\mubB$ with the shape tuple $\mbm$ and layout tuple $\mbtau$ tuples, both of length $p$.
%In this work, it permutes the shape and layout tuple simultaneously without changing the element ordering and without copying tensor elements.
%The operation $\rho$ is defined by a permutation tuple $\mbrho = (\rho_1,\dots,\rho_p)$ that defines elements of $\mbm$ and $\mbtau$ with $m_r = n_{\rho_r}$ and $\tau_r = \pi_{\rho_r}$, respectively.
%\vspace{-1em}


