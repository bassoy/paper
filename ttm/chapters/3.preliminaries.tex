\section{Background}
\label{sec:preliminaries}

\subsubsection{Notation}
\label{sec:preliminaries:notation}
An order-$p$ tensor is a $p$-dimensional array \cite{lim:2017:hypermatrices} where tensor elements are contiguously stored in memory. % \cite{lee:2018:fundamental}
We write $a$, $\mba$, $\mbA$ and $\mubA$ in order to denote scalars, vectors, matrices and tensors. 
If not otherwise mentioned, we assume $\mubA$ to have a tensor order that is greater than $2$.
The $p$-tuple $\mbn$ with $\mbn = (n_1,n_2,\dots,n_p)$ will be referred to as a dimension tuple with $n_r>1$.
We will use round brackets $\mubA(i_1,i_2,\dots,i_p)$ or $\mubA(\mbi)$ to denote a tensor element where $\mbi = (i_1,i_2,\dots,i_p)$ is a multi-index.
A subtensor is denoted by $\mubA'$ and references elements of a tensor $\mubA$.
They are specified with $p$ index ranges and form a selection grid.
In this work, the index range shall either address all indices of a given mode or a single element that are given by single indices $i_r$ with $1 \leq r \leq p$. % denoted by $\colon$ 
Elements $n_r'$ of a subtensor's dimension tuple $\mbn'$ are therefore $n_r$ if all indices of mode $r$ are selected and $1$ otherwise.
We will annotate subtensors using only their non-unit modes such as $\mubA_{u,v,w}'$ where $n_u > 1$,$n_v > 1$ and $n_w >1$ and $1 \leq u \neq v \neq w \leq p$.
It is sufficient to only provide non-unit modes as the remaining single indices correspond to the loop induction variables of the following algorithms.
%todo: what is the order? p' <? p where all non-unit dimensions are counted and what happens with the layout?
A subtensor is called a slice $\mubA_{u,v}'$ if the full range selection of $\mubA$ occurs with only two modes.
A fiber $\mubA_u'$ is a tensor slice with only one dimension greater than $1$.
\vspace{-1.5em}

\subsubsection{Linear Tensor Layouts}
\label{sec:preliminaries:layout}
We use a layout tuple $\mbpi \in \bbN^p$ to encode all linear tensor layouts including the first-order or last-order layout.
They contain permuted tensor modes whose priority is given by their index.
For instance, the first- and last-order storage formats are given by $\mbpi_F = (1,2,\dots,p)$ and $\mbpi_{L} = (p,p-1,\dots,1)$.
An inverse layout tuple $\mbpi^{-1}$ is defined by $\mbpi^{-1}(\mbpi(k)) = k$.
Given a layout tuple $\mbpi$ with $p$ modes, the $\pi_r$-th element of a stride tuple is given by $w_{\pi_r} = \prod_{k=1}^{r-1} n_{\pi_k}$ for $1 < r \leq p$ and $w_{\pi_1} = 1$.
Tensor elements of the $\pi_1$-th mode are contiguously stored in memory.
The location of tensor elements is determined by the tensor layout and the layout function.
For a given tensor layout and stride tuple, a layout function $\lambda_{\mbw}$ maps a multi-index to a scalar index with $\lambda_{\mbw}(\mbi) = \sum_{r=1}^p w_r (i_r-1)$.
With $j = \lambda_{\mbw} (\mbi)$ being the relative memory position of an element with a multi-index $\mbi$, reading from and writing to memory is accomplished with $j$ and the first element's address of $\mubA$.
\vspace{-1.5em}

\subsubsection{Non-Modifying Flattening and Reshaping}
\label{sec:preliminaries:flattening.reshaping}
The flattening operation $\varphi_{r,q}$ transforms an order-$p$ tensor $\mubA$ to another order-$p'$ view $\mubB$ that has different a shape $\mbm$ and layout $\mbtau$ tuple of length $p'$ with $p' = p-q+r$ and $1 \leq r < q \leq p$.
It is related to the tensor unfolding operation as defined in \cite[p.459]{kolda:2009:decompositions} but neither changes the element ordering nor copies tensor elements.
Given a layout tuple $\mbpi$ of $\mubA$, the flattening operation $\varphi_{r,q}$ is defined for contiguous modes $\mbhpi = (\pi_r,\pi_{r+1}, \dots, \pi_{q})$ of $\mbpi$.
Let $j = 0$ if $k \leq r$ and $j = q-r$ otherwise for $1 \leq k \leq p'$.
%Let also $s_k = 0$ if $\pi_k < \max(\mbpi_{r,q})$ and $s_k = q-r$ if $\pi_{k+j} > \max(\mbpi_{r,q})$.
Then the resulting layout tuple $\mbtau = (\tau_1,\dots,\tau_{p'})$ of $\mubB$ is given by $\tau_r = \min(\mbpi_{r,q})$ and $\tau_{k} = \pi_{k+j} + s_k$ if $k \neq r$ where $s_k = \left| \{ \pi_i \mid \pi_{k+j} > \pi_i \wedge \pi_i \neq \min(\mbhpi) \wedge r \leq i \leq p \} \right|$.
Elements of the corresponding shape tuple $\mbm$ are given by $m_{\tau_r} = \prod_{k=r}^q n_{\pi_k}$ and $m_{\tau_k} = n_{\pi_{k+j}}$ if $k \neq r$.

The reshaping operation $\rho$ transforms an order-$p$ tensor $\mubA$ to another order-$p$ tensor $\mubB$ with different shape $\mbm$ and layout $\mbtau$ tuples of length $p$.
In this work, it permutes the shape and layout tuple simultaneously without changing the element ordering and without copying tensor elements.
The operation $\rho$ uses a permutation tuple $\mbrho = (\rho_1,\dots,\rho_p)$ to only modify shape and layout tuples.
Elements of the resulting shape tuple $\mbm$ and the layout tuple $\mbtau$ are given by $m_r = n_{\rho_r}$ and $\tau_r = \pi_{\rho_r}$, respectively.
\vspace{-1.5em}

\subsubsection{Tensor-Matrix Multiplication (TTM)}
Let $\mubA$ and $\mubC$ be order-$p$ tensors with shapes $\mbn_a = (n_1,\dots,n_q,\dots,n_p)$ and $\mbn_c =(n_1,\dots,n_{q-1},m,n_{q+1},\dots,n_p)$. 
Let $\mbB$ be a matrix of shape $\mbn_b = (m,n_q)$.
A mode-$q$ TTM is denoted by $\mubC = \mubA \times_q \mbB$ where an element of $\mubC$ is given by
\begin{equation}
\label{equ:tensor.matrix.multplication}
\mubC(i_1, \dots, i_{q-1}, j, i_{q+1}, \dots, i_p) = \sum_{i_q=1}^{n_q} \mubA(i_1, \dots, i_q, \dots, i_p) \cdot \mbB(j,i_q)
\end{equation}
with $1 \leq i_r \leq n_r$ and $1 \leq j \leq m$.
%Similar to the tensor-vector multiplication, the multiplication consists of multiple inner productd of a fiber of $\mubA$ and $\mbb$.% with $1 \leq i_r \leq n_r$ and $1 \leq r \leq p$.
The mode $q$ is the \tit{contraction} \tit{mode} of the TTM  with $1 \leq q \leq p$.
The tensor-matrix multiplication generalizes the computational aspect of the two-dimensional case $\mbC = \mbB \cdot \mbA$ if $p=2$ and $q=1$.
Its arithmetic intensity is equal to that of a matrix-matrix multiplication and is not memory-bound.
%Categorized in \cite{dinapoli:2014:towards.efficient.use} as an operation of the tensor contraction class 2, its computation is likely to be limited by the memory bandwidth.
In the following, we assume that the tensors $\mubA$ and $\mubC$ have the same tensor layout $\mbpi$. 
Elements of matrix $\mubB$ can stored in either the column-major or row-major format.

