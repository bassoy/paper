\section{Introduction}
\label{sec:introduction}
Tensor computations are found in many scientific fields such as computational neuroscience, pattern recognition, signal processing and data mining \cite{karahan:2015:tensor, papalexakis:2017:tensors}.
Tensors representing large amount of multidimensional data are decomposed and analyzed with the help of basic tensor operations \cite{lee:2018:fundamental, kolda:2009:decompositions}. 
%While tensor contraction with arithmetic intensity has been discussed, class 2 tensor operations 
The decomposition and analysis led to the development and analysis of high-performance kernels for tensor contractions.
%todo: check above sentence.
In this work, we present and analyze a high-performance algorithm for the tensor-matrix multiplication that is used in many numerical algorithms such as the alternating least squares method \cite{lee:2018:fundamental, kolda:2009:decompositions}.
It is a compute-bound tensor operation and has the same arithmetic intensity as a matrix-matrix multiplication which can almost reach the practical peak performance of a computing machine.
%%with free tensor indices, most implementations of the above mentioned approaches reach near peak performance of the computing machine\cite{springer:2018:design, matthews:2018:high,shi:2016:tensor.contraction}. 

There has been three main approaches for implementing tensor contractions.
The Transpose-Transpose-\tf{gemm}-Transpose (TGGT) approach reorganizes (flattens) tensors in order to perform a tensor contraction with an optimized matrix-matrix multiplication (\tf{gemm}) implementation \cite{bader:2006:algorithm862, solomonik:2013:cyclops}.
Implementations of the GETT method are based on high-performance \tf{gemm}-like algorithms \cite{springer:2018:design, matthews:2018:high}.
A different method is the LOG approach in which BLAS are utilized with multiple tensor slices or subtensors if possible \cite{dinapoli:2014:towards.efficient.use, li:2015:input, shi:2016:tensor.contraction}.
Implementations of the LOG and TTGT approaches are in general easier to maintain and faster to port than GETT implementations which might need to adapt vector instructions or blocking parameters according to a processor's microarchitecture.
%todo: Compiler-based approaches like as described in \cite{gareev:2018:high} 

%Our work is motivated by the fact that LOG-based implementations of the tensor-matrix and tensor-vector multiplication are similar. 
To our best knowledge, we are the first to combine the approach described in \cite{bassoy:2019:ttv} with the findings on tensor slicing in \cite{li:2015:input} and to propose fast in-place algorithms that are layout-oblivious.
%todo: explain layout-oblivious
Our algorithms compute the tensor-matrix product in parallel with highly efficient \tf{gemm} or \tf{gemm\_batch} implementations.
They support dense tensors with any order, dimensions and any linear tensor layout including the first- and the last-order storage formats for any contraction mode all of which can be runtime variable.
Input and output tensors do not need to be transposed or flattened into two-dimensional matrices.
The parallel versions of the recursive base algorithm execute fused loops in parallel and are able to fully utilize a processor's compute units.
Despite the generality of our approach, every proposed algorithm can be implemented with less than 100 lines of \tf{C++} code where the algorithmic complexity is reduced by the BLAS implementation and the corresponding selection of subtensors or tensor slices.
We have provided an open and free reference \tf{C++} implementation of all algorithms and a python interface for convenience.
While we have used Intel's MKL for our benchmarks, the user is free to choose any other library that provides the BLAS interface.

The following analysis quantifies the impact of the tensor layout, the tensor slicing method and parallel execution of slice-matrix multiplications with varying contraction modes.
The runtime measurements of our implementations are compared with those presented in \cite{springer:2018:design, matthews:2018:high} including Libtorch and Eigen.
In summary, the main findings of our work are:
\begin{itemize}
	\item 
	%
	A tensor-matrix multiplication can be implemented by an in-place algorithm with $1$ \tf{gemv} and $7$ \tf{gemm} calls, supporting all combinations of contraction mode, tensor order and dimensions for any linear tensor layout.
	\item 
	Our algorithm with variable loop fusion and parallel slice-matrix multiplications is on average $17$\% faster than Intel's \tf{gemm\_batch} when the contraction and leading dimensions of the tensors are greater than $256$.
	\item 
	Our algorithms are layout oblivious.
	The fastest implementations achieve at least a median throughput of \tq for any linear tensor layout.
	\item
	Our fastest algorithm computes the tensor-matrix multiplication on average, by at least $14.05$\% and up to a factor of $3.79$ faster than other state-of-the art library implementations, including LibTorch and Eigen.
\end{itemize}

The remainder of the paper is organized as follows. 
Section~\ref{sec:related} presents related work.
Section~\ref{sec:preliminaries} introduces some notation on tensors and defines the tensor-matrix multiplication.
Algorithm design and methods for slicing and parallel execution are discussed in Section~\ref{sec:design}.
Section~\ref{sec:experimental.setup} describes the test setup. Benchmark results are presented in Section \ref{sec:results}.
Conclusions are drawn in Section~\ref{sec:conclusion}.